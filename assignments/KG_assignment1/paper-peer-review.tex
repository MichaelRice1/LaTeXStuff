\documentclass[a4paper,9pt]{scrartcl}
\usepackage{amssymb, amsmath} % needed for math
\usepackage[utf8]{inputenc}   % this is needed for umlauts
\usepackage[USenglish]{babel} % this is needed for umlauts
\usepackage[T1]{fontenc}      % this is needed for correct output of umlauts in pdf
\usepackage[margin=2.5cm]{geometry} %layout
\usepackage[hyperfootnotes=false, pdfborder={0 0 0}, colorlinks=false]{hyperref}
\usepackage{color}
\usepackage{framed}
\usepackage{enumerate}  % for advanced numbering of lists
\usepackage{csquotes}   % for enquote

\setlength{\parskip}{0.25em}  % Increase space between paragraphs
\setlength{\parindent}{3em}  % Change the indent size

\newcommand\titletext{Review of: \\ 
\textbf{"Advancing Robotic Perception with Perceived-Entity Linking"} \\ 
\textit{ISWC 2024, Volume 2, Pg 192-209} \\[0.5em] 
\textit{Adamik et al.}}
\title{\titletext}
\author{Michael Rice}
\usepackage{microtype}

\begin{document}

\maketitle
\section{Paper Summary}
This paper discusses a novel approach to robotic perception,  known as Perceived Entity Linking or PEL, which operates by 
leveraging the Semantic Web to link visually perceived entities to uniquely identifiable entries in target knowledge repositories. The authors propose 
a fascinating approach, attempting to form the aforementioned link between what the robot perceives and what it knows using the raw sensory data 
captured by the robot's sensor array. \cite{adamikAdvancingRoboticPerception2024}. This publication draws on techniques from both the fields of 
Computer Vision and Natural Language Processing to remedy several challenges in robotic perception, including entity typing (both 
with and without entity attributes i.e. size, colour etc.) as well as entity recognition and entity linking. 

The primary technique from the field of computer vision deployed in this paper is object detection, in the form of the YOLO (You Only Look Once) 
algorithm \cite{redmonYouOnlyLook2016}. Additionally, two techniques, inspired by the field of Natural Language Processing (NLP), are employed: Candidate 
Generation and Disambiguation. The Candidate Generation process is utilized to propose a set of potential perceived entities based on results from the
object detection algorithm. For instance, if the label 'Orange' is returned, WordNet \cite{WordNetLexicalDatabase} is utilized to provide alternative 
variations or interpretations of the label, such as 'orange' referring to the fruit or 'Orange' referring to the city in California. Following on from this, 
the Disambiguation process seeks to refine this list of candidates, reducing it to either a single entity or no entity at all. These techniques combine to 
form a pipeline that operates under the following headings - observation, (entity) recognition and (entity) linking \cite{adamikAdvancingRoboticPerception2024} in 
an attempt to equip robots with a form of 'commonsense' environmental knowledge. 




\section{Expected Impact of the Paper}

I believe that the future impact of the work undertaken in this paper could potentially be significant, particularly in the field of general-purpose robotics. 
Enhancing 'common-sense' knowledge in robotic perception systems is the key to increasing the generalizability of such systems. The discussed concept of PEL \cite{adamikAdvancingRoboticPerception2024}, which will likely be 
central to any potential future impact this paper may have, is enabled by a variety of individual building blocks, which combine to achieve its overall goal. 

First, the linking of sensory data to known entities in knowledge bases, could make use of the RDF model \cite{ResourceDescriptionFramework}, which uses subject-predicate-object triples to create
links between pieces of information. For example, if an object detection algorithm is fine-tuned to detect car manufacturer, and a Volvo car is being perceived by the robot, one such triple that could represent the situation would be Volvo - rdf:type - Car
, as the 'Car' entity is linked to the 'Volvo' entry in the knowledge graph as a subclass. Furthermore, the fact that RDF is a standardized and machine-readable framework could potentially 
allow for the sharing, understanding and processing of the same information by multiple different systems without issue. RDF also supports ontology languages such as OWL \cite{WebOntologyLanguage}, 
which provide a structured framework for helping robots understand entity types and their relationships. For example, OWL can define that a leaf is a type of plant, enabling more accurate linking of s
ensory data to recognized entities and improving overall entity classification.


Secondly, SPARQL \cite{SPARQL11Query} and Turtle \cite{RDF11Turtle} could potentially be in use here as a framework for the efficient querying and serialization of linked knowledge. These two concepts are 
critical to the lowering of delays in the linking process, an important factor in the overall success and long-term implications of this work. 



\section{Primary Weakness}

The main weaknesses of this implementation are twofold. First, although the peak achieved accuracy of the PEL process on the sample datasets, (69\%), is 
commendable for an initial attempt, it falls short of the standards required for real-world deployment. Second, both the size and quality of the knowledge 
base present significant challenges to any potential real-world deployment. An excessively large knowledge base can negatively impact real-time performance, while a low-quality knowledge base 
can compromise linking accuracy.

In order to first tackle the issue of linking accuracy, a number of solutions could be proposed. The first could 
augment the Disambiguation process through the use of a ViT (Vision Transformer) \cite{dosovitskiyImageWorth16x162021}. A ViT could be used to provide
general scene context to the robot, enabling a more accurate and dependable process. 

Secondly, to address the issues relating to the knowledge base itself, a three-part hybrid solution could be proposed to solve the size issues, in the form of 
dynamically-constructed\cite{chenConstructingDynamicKnowledge2022}, context-aware \cite{panContextawareEntityTyping2021} and domain-specific  
\cite{abu-salihDomainspecificKnowledgeGraphs2021} sub-graphs. As large knowledge bases can negatively impact real-time performance, sub-graphs could 
be constructed in real-time based on the scene's context, which could in turn be extracted using a variety of techniques, not limited to; object detection 
algorithms, (semantic) SLAM algorithms \cite{liSDSLAMSemanticSLAM2024} or Vision Transformer implementations \cite{dosovitskiyImageWorth16x162021}. This would 
limit the knowledge graph to only what is currently relevant to the robot's surroundings, reducing search and retrieval delays. To actually extract entities from the graph 
that are related to the scene context, SPARQL \cite{SPARQL11Query} queries could again be used. For example, if the scene context is understood to be an object located in a kitchen, 
queries could be used to extract all entities with a sample 'foundIn' relationship to 'kitchen'. 
An example being: 
\begin{center}
\verb|SELECT * WHERE {?subject foundIn "kitchen".}|
\end{center}


Finally, the issue of knowledge graph quality is more challenging to address, as it relies on externally available information.
However, a solution to this issue could potentially involve the use of in-depth, domain-specific knowledge bases alongside more general, publicly available knowledge bases.



\section{Comparison to Other Published Works}


\subsection{Work 1 : "Know Rob 2.0 — A 2nd Generation Knowledge Processing Framework for Cognition-Enabled Robotic Agents" - Beetz et al.}

The first paper I will use for comparison to this work is "Know Rob 2.0" by Beetz et al., a second generation framework created to tackle knowledge representation and reasoning 
challenges for robotic agents. In contrast to the objective of the paper under review, where the goal was to link perceived entities to a predefined knowledge base, the goal 
of "Know Rob 2.0" is to 'bridge the gap' between user instructions such as 'open the fridge' and the detailed robotic actions/motions required to perform the task
\cite{beetzKnowRob202018}. This process aims to incorporate environmental specifics into the robotic agent's decision-making process, such as the weight of an object
that is to be picked up or the orientation it should be held in. In essence, the goal of this implementation is again to provide the robots with a degree of 'common-sense 
knowledge' about it's surroundings and how to use this knowledge to its advantage.

To further extend Beetz's work, PEL \cite{adamikAdvancingRoboticPerception2024}, as discussed by Adamik et al., could be integrated into "Know Rob 2.0" in an attempt to 
more accurately determine/understand objects in the robot's environment, which are necessary to perform the task at hand. This could potentially enhance their ability to 
perform complex tasks by augmenting the decision-making process with detailed object attributes acquired from the accompanying knowledge base.


\subsection{Work 2 : "Grounding Robot Sensory and Symbolic Information Using the Semantic Web" - Stanton and Williams}

The second paper I will use for comparison is "Grounding Robot Sensory and Symbolic Information Using the Semantic Web" by Stanton and Williams. This paper concerns
a theoretical discussion of symbol grounding (linking internal symbols used in decision-making to the real-world phenomena they refer to) \cite{stantonGroundingRobotSensory2004}
using ontologies (frameworks for knowledge representation and machine understanding of data and relationships) originally designed for the Semantic Web. 
In comparison to Adamik et al.'s and Beetz et al.'s paper, Stanton and Williams' work is an earlier and more foundational foray into understanding how to map raw sensory data to 
real-world information. As such, it is interesting to compare it to more recent works. However, Stanton and Williams provide no details on an implementation, offering only a 
discussion of the broader concepts and challenges, which sets their work apart from the other two papers under review.

In their paper, Stanton and Williams focus on using a soccer-playing robot as a canvas to illustrate their framework and approach. They explore the
use of ontologies in the grounding process (e.g. linking an orange blob in the robot's vision to a soccer ball, due to known properties of the ball.) By doing so, robots are
able to build complex, internal models of their environments, allowing them to reason and plan as well as predict future states \cite{stantonGroundingRobotSensory2004}.
The role of ontologies in this is to aid in the description of entities is terms of their sensory data, which in turn allows a connection between the 'Perception Layer' and 
the 'Symbolic Layer' \cite{stantonGroundingRobotSensory2004}.





\rule{\linewidth}{0.2mm}

\bibliographystyle{plain}
\bibliography{/mnt/c/Users/athen/Desktop/MAI_Sem2/KnowledgeGraphs/Assignment1/KG_Assignment1.bib}


\end{document}
