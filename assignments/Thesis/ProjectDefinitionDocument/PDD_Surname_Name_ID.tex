\documentclass[11pt]{article}
\usepackage[margin=0.4in]{geometry}
\usepackage[all]{xy}

\usepackage{amsmath,amsthm,amssymb,color,latexsym,graphicx}
\usepackage{geometry}        
\geometry{a4paper, top=1cm}    

\usepackage[scaled]{helvet}
\renewcommand\familydefault{\sfdefault} 
\usepackage[T1]{fontenc}

\usepackage{tabto}

\usepackage[numbers]{natbib}

\linespread{1.0}

\pagenumbering{gobble}

\newcommand{\sect}[1]{\begin{center}\textbf{#1}\end{center}}

\begin{document}
	
\noindent Student name: Michael Rice \tabto{8.5cm} Supervisor contact name: Waqar Shahid Qureshi \\
\noindent Student number: 20347541 \tabto{8.5cm} Supervisor contact email: waqarshahid.qureshi@universityofgalway.ie 
\vspace{0.1cm}

\sect{From Sight to Insight: Harnessing Gaze and LLMs to provide Real-Time Educational Assistance}

\sect{Introduction}

Advancements in wearable technology and artificial intelligence have opened new possibilities for real-time educational assistance. 
This project leverages Meta’s Project Aria \cite{engelProjectAriaNew2023} glasses, object detection, and large language models (LLMs) 
to create a gaze-based system that provides users with contextual information about the objects they focus on. By integrating 
egocentric gaze tracking with AI-driven content generation, this system aims to enhance learning in various domains.

\sect{Background Research or Context}

A wide variety of regions will require extensive background research in order to complete this project. Firstly, I will need to 
understand the hardware, the Project Aria Glasses, that I will be using, their usefulness to my project and the proprietary file system that comes with them, VRS.
Following this, I will research object detection options, along with each of their strengths and 
weaknesses.
Finally, research into Large Language Models (LLMs) will be required. This will involve understanding the various open-source and 
viable LLM options available and how they can be refined for my specific use case.

\sect{Proposed Project}
This project will develop a real-time educational aid that identifies objects in a user's field of view and provides relevant 
information using LLMs. The system will rely on Project Aria’s eye tracking cameras and gaze estimation to determine where the user 
is looking. An object detection algorithm (e.g., YOLO \cite{redmonYouOnlyLook2016} or Detectron2) will then classify the object, and an LLM will generate 
concise, context-aware descriptions or explanations.

The key components of the system include: Gaze Tracking, Object Detection and Recognition, and Contextual Information Retrieval.

This tool has applications in education, professional training, and even specialized fields such as agriculture, where it can provide
detailed insights into plant species or farming equipment based on gaze detection.


\sect{Timeline}

February - March 2025: Literature Research and Getting Project Aria Glasses up and running. \\

April - May 2025: Creating base implementations of each algorithmic aspect.  \\

June - July 2025: Move from base implementations to final models and integrating them all together.  \\

August 2025: Tidying up final implementation and writing thesis. \\





\bibliographystyle{IEEEtranN}

\renewcommand{\refname}{{\normalsize \sect{References}}}

% >>>>> Unless a different referencing style is stipulated by your supervisor, it is strongly recommended to use IEEE style: https://libguides.ncirl.ie/referencingandavoidingplagiarism/ieee

% It is strongly recommended to use a BibTeX file (.bib) for the bibliography, as this makes it much easier to
% collect references and to reuse them later in other LaTeX documents: 
\bibliography{/mnt/c/Users/athen/Desktop/Thesis/Thesis.bib}

% However, alternatively references can also be listed manually like this (not recommended):
%\begin{thebibliography}{10}	
%\bibitem{Linde1980} Y. Linde, A. Buzo and R.M. Gray, ``An algorithm for vector quantizer design,'' IEEE Transactions on Communications, vol. 28, pp. 84-95, January 1980. 
%\bibitem{Fanty1993} M. Fanty, P. Schmid, and R. Cole, ``City name recognition over the telephone,'' in Proceedings Third International Conference on Acoustics, Speech and Signal Processing, Minneapolis, USA, pp. 549-552, 1993. 
%\end{thebibliography}

\end{document}